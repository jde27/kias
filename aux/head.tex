\usepackage{amsthm,amsmath,amsfonts,amscd}
\usepackage{tikz}
\usepackage{verbatim}
\usepackage{caption}
\usepackage{pgfplots}
\usetikzlibrary{decorations.pathreplacing}
\usetikzlibrary{decorations.markings}
\usetikzlibrary{decorations.pathmorphing}
\tikzset{snake it/.style={decorate, decoration=snake}}
\tikzset{->-/.style={decoration={
              markings,
              mark=at position .5 with {\arrow{>}}},postaction={decorate}}}
\tikzset{->>-/.style={decoration={
              markings,
              mark=at position .5 with {\arrow{>>}}},postaction={decorate}}}
\tikzset{->>>-/.style={decoration={
              markings,
              mark=at position .5 with {\arrow{>>>}}},postaction={decorate}}}
\tikzset{->>>>-/.style={decoration={
              markings,
              mark=at position .5 with {\arrow{>>>>}}},postaction={decorate}}}
\newcommand{\conic}[1]{
\begin{scope}[shift={#1}]
\filldraw[fill=white,draw=none] (-0.5,-0.5) to[out=45,in=-45] (-0.5,0.5) -- (0.5,0.5) to[out=-135,in=135] (0.5,-0.5) -- (-0.5,-0.5);
\filldraw[fill=white] (0,0.5) circle [x radius=0.5,y radius=0.25];
\filldraw[fill=white,dashed] (-0.5,-0.5) arc [x radius=0.5,y radius=0.25,start angle=180,end angle=0];
\filldraw[fill=white] (-0.5,-0.5) arc [x radius=0.5,y radius=0.25,start angle=180,end angle=360];
\draw (-0.5,-0.5) to[out=45,in=-45] (-0.5,0.5);
\draw (0.5,-0.5) to[out=135,in=-135] (0.5,0.5);
\end{scope}
}
\newcommand{\nodalconic}[1]{
\begin{scope}[shift={#1}]
\filldraw[fill=white,draw=none] (-0.5,-0.5) -- (0,0) -- (0.5,-0.5) -- cycle;
\filldraw[fill=white,draw=none] (-0.5,0.5) -- (0,0) -- (0.5,0.5) -- cycle;
\filldraw[fill=white] (0,0.5) circle [x radius=0.5,y radius=0.25];
\filldraw[fill=white,dashed] (-0.5,-0.5) arc [x radius=0.5,y radius=0.25,start angle=180,end angle=0];
\filldraw[fill=white] (-0.5,-0.5) arc [x radius=0.5,y radius=0.25,start angle=180,end angle=360];
\draw (-0.5,-0.5) -- (0.5,0.5);
\draw (-0.5,0.5) -- (0.5,-0.5);
\end{scope}
}
\newcommand{\circdash}[4]{
\begin{scope}[shift={#1},scale=#2,rotate=#4]
\draw[#3,dashed] (-1,0) arc [x radius=1,y radius=0.5,start angle=180,end angle=0];
\draw[#3] (1,0) arc [x radius=1,y radius=0.5,start angle=0,end angle=-180];
\end{scope}
}
\definecolor{dg}{HTML}{228B22}
\usepackage{bm}
\usepackage{pinlabel}
\usepackage{parskip}
\usepackage{ragged2e}
\usepackage{float}
\usepackage{subfig}
\usepackage[utf8]{inputenc}
\newcommand{\CC}{\mathbf{C}}
\newcommand{\QQ}{\mathbf{Q}}
\newcommand{\RR}{\mathbf{R}}
\newcommand{\UU}{\mathbf{U}}
\newcommand{\XX}{\mathbf{X}}
\newcommand{\YY}{\mathbf{Y}}
\newcommand{\ZZ}{\mathbf{Z}}
\newcommand{\cp}[1]{\mathbf{CP}^{#1}}
\newcommand{\rp}[1]{\mathbf{RP}^{#1}}
\newcommand{\OP}[1]{\mathrm{#1}}
\newcommand{\matr}[4]{\left(\begin{array}{cc}#1 & #2\\ #3 & #4\end{array}\right)}
\newcommand{\mthrthr}[9]{\left(\begin{array}{ccc}#1 & #2 & #3\\ #4 & #5 & #6\\ #7 & #8 & #9\end{array}\right)}
\newcommand{\vect}[2]{\left(\begin{array}{c}#1\\#2\end{array}\right)}
\newcommand{\vthr}[3]{\left(\begin{array}{c}#1\\#2\\#3\end{array}\right)}
\newcommand{\TO}[3]{#1\stackrel{#2}{\longrightarrow}#3}
\newenvironment{Proof}{\begin{proof}}{\end{proof}\ignorespacesafterend}
\begingroup
    \makeatletter
    \@for\theoremstyle:=definition,remark,plain\do{%
        \expandafter\g@addto@macro\csname th@\theoremstyle\endcsname{%
            \addtolength\thm@preskip\parskip
            }%
        }
\endgroup
\usepackage{graphicx}
\usepackage[capitalise,noabbrev]{cleveref}
\newtheorem{Theorem}{Theorem}[section]
\newtheorem*{Theorem*}{Theorem}
\newtheorem{Lemma}[Theorem]{Lemma}
\newtheorem{Corollary}[Theorem]{Corollary}
\newtheorem{Claim}[Theorem]{Claim}
\newtheorem{Proposition}[Theorem]{Proposition}
\theoremstyle{remark}
\newtheorem{Remark}[Theorem]{Remark}
\theoremstyle{definition}
\newtheorem{Definition}[Theorem]{Definition}
\newtheorem{Example}[Theorem]{Example}
\newtheorem{Exercise}[Theorem]{Exercise}
\crefname{Theorem}{Theorem}{Theorems}
\Crefname{Theorem}{Theorem}{Theorems}
\crefname{Lemma}{Lemma}{Lemmas}
\Crefname{Lemma}{Lemma}{Lemmas}
\crefname{Corollary}{Corollary}{Corollaries}
\Crefname{Corollary}{Corollary}{Corollaries}
\crefname{Claim}{Claim}{Claims}
\Crefname{Claim}{Claim}{Claims}
\crefname{Proposition}{Proposition}{Propositions}
\Crefname{Proposition}{Proposition}{Propositions}
\crefname{Remark}{Remark}{Remarks}
\Crefname{Remark}{Remark}{Remarks}
\crefname{Definition}{Definition}{Definitions}
\Crefname{Definition}{Definition}{Definitions}
\crefname{Example}{Example}{Examples}
\Crefname{Example}{Example}{Examples}
\crefname{Exercise}{Exercise}{Exercises}
\Crefname{Exercise}{Exercise}{Exercises}
\setcounter{tocdepth}{1}
